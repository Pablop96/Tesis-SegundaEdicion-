\chapter*{Prefacio}
   % \thispagestyle{empty}

\lettrine[lines=9] {\initfamily  \selectfont C}{uando} iba en el tercer semestre de la licenciatura, tomé el curso \textit{Gráficas y Juegos} con la maestra Laura Pastrana, en la Facultad de Ciencias. Era tal el amor con el que Laura daba su clase que me lo transmitió inmediatamente, y supe desde ese momento que quería dedicarme a la Teoría de Gráficas. Es una rama de las Matemáticas tan elegante y bella, pero al mismo tiempo intrigante. Después de varias experiencias, me di cuenta que me sentía más atraído por las Matemáticas Aplicadas y deseaba estudiar algo que \textit{aplicado} que al mismo tiempo involucrara a las gráficas. 


Conocí al Dr. Gilberto Calvillo a finales de junio del 2017, mientras él impartía una conferencia en la Escuela de Verano de Cuernavaca que suele organizarse anualmente. La plática se titulaba \textit{Los Teoremas del Problema de Asignación Óptima} y recuerdo perfectamente que él hablaba de cómo ciertos problemas del Álgebra Lineal pueden ser modelados y resueltos en la Teoría de Gráficas. Quedé maravillado: ¿cómo era posible que esas dos ramas de las Matemáticas (aparentemente sin relación alguna) tuvieran tanto en común? Fue como si un horizonte infinito se extendiera ante mis ojos y yo quería ser parte de eso.


Decidí, en diciembre de ese mismo año pedirle al Dr. Calvillo que fuera mi director de tesis y, para mi fortuna, aceptó. En enero de 2018, él me recomendó (que a su vez ya había sido sugerido por el Dr. Jesús López Estrada) el tema que he plasmado en este trabajo.


Mi tesis fue fruto de un esfuerzo de, más o menos, un año y medio: de junio a diciembre de 2018 me dediqué a leer varios libros y comprender los conceptos y los teoremas; y en 2019 seguí leyendo y escribiendo mi trabajo, mientras lo complementaba trabajando como ayudante de profesor en la Facultad de Ciencias y siendo becario del Instituto de Matemáticas.


Realmente esta tesis fue terminada a finales de enero de 2020 para después ser leída y corregida por mis sinodales. Pero nadie se imaginaría que a mediados de marzo de ese mismo año la pandemia de la enfermedad COVID-19 azotaría al mundo entero, interrumpiendo de golpe las actividades que requerían la presencia humana y encerrándonos de forma obligada a una larga cuarentena. 


Probablemente desde los eventos de la Segunda Guerra Mundial de los años 40 del siglo XX, la humanidad entera no se había visto envuelta en un evento tan traumático y de tal magnitud como el de esta pandemia mundial.  Inevitablemente, este acontecimiento ha cambiado el mundo para siempre. Desde una gran crisis económica, provocando que millones de personas perdieran sus negocios y sus empleos, hasta el lamentable fallecimiento de numerosas vidas humanas que, quizás, en otras circunstancias estarían hoy con nosotros. 


Por otro lado, ha supuesto una revolución tecnológica como nunca se había visto. Paradójicamente, a pesar de estar separados por el confinamiento, estamos más conectados que antes: las videollamadas, las videoconferencias, las clases virtuales, hacer transacciones bancarias, pedir comida, comprar la despensa, solicitar un taxi, etc., y todo desde la comodidad del hogar y a través de un celular. Lo anterior nos ha permitido afrontar esta problemática de formas que hace tan sólo diez años no hubiéramos podido. Es en este contexto en el que escribo estas palabras y en el que esta tesis queda registrada. Los exámenes profesionales por el momento son en línea y así será el mío en el 2021.


No me cabe duda de que estamos viviendo en una de las mejores épocas de la humanidad: desde tecnología, educación y comodidades hasta la inclusión de minorías que, en otros tiempos, fueron calladas. Tenemos a nuestro alcance hoy lo que los reyes, conquistadores y emperadores más opulentos del pasado ni se hubieran imaginado. Que esta sea de las mejores épocas nos significa que sea perfecta. Todavía nos quedan retos que superar como el cambio climático, la difusión de noticias falsas, el mal uso de los datos, el arribo de personas de dudosa capacidad a los puestos más altos de poder, la ignorancia y, en general, los problemas existenciales del ser humano moderno.

Sin embargo, en cada momento tenemos la responsabilidad de elegir hacia dónde queremos dirigirnos. La decisión está en nuestras manos en cada momento y en cada pequeño detalle de la vida diaria: somos capaces de actuar a favor de nuestro progreso como seres humanos y de nuestra sociedad, o podemos quedarnos cruzados de brazos y no hacer nada al respecto. Como dijo el filósofo italiano Giovanni Pico Della Mirandola en su Discurso sobre la Dignidad Humana:

\vspace{0.5cm}
``\textit{El Óptimo Artífice […] tomó entonces al hombre así concebido, obra de la naturaleza indefinida y, poniéndolo en el centro del mundo, le habló de esta manera: No te di, Adán, ni un puesto determinado ni un aspecto propio ni función alguna que te fuera peculiar, con el fin de que aquel puesto, aquel aspecto, aquella función por la que te decidieras, los obtengas y conserves según tu deseo y designio. La naturaleza limitada de los otros se halla determinada por las leyes que yo he dictado. La tuya, tú mismo la determinarás sin estar limitado por barrera ninguna, por tu propia voluntad, en cuyas manos te he confiado. Te puse en el centro del mundo con el fin de que pudieras observar desde allí todo lo que existe en el mundo. No te hice ni celestial ni terrenal, ni mortal ni inmortal, con el fin de que –casi libre y soberano artífice de ti mismo- te plasmaras y te esculpieras en la forma que te hubieras elegido. Podrás degenerar hacia las cosas inferiores que son los brutos; podrás -de acuerdo con la decisión de tu voluntad- regenerarte hacia las cosas superiores que son divinas}”.

\vspace{1 cm}
En esta tesis se hablará de una parte de la Teoría Algebraica de Gráficas, que es la rama de las matemáticas que se ocupa de aplicar los métodos algebraicos en problemas que conciernen a las gráficas. Estos métodos pueden incluir, por ejemplo, el uso de la teoría de grupos para estudiar invariantes en gráficas. Pero también, como ya mencionamos al principio, el Álgebra Lineal juega un papel importante en el estudio de la Teoría de Gráficas y en este trabajo nos enfocamos en estudiar esa relación.


El texto base de esta tesis es el libro escrito por J. A. Bondy y U. S. R. Murty titulado \textit{Graph Theory} \cite{Bondy}. Además de considerarlo modernos, nos  pareció conveniente  su notación y su terminología. El lector podrá hallar en \cite{Deo} y \cite{Seshu} un desarrollo mucho más detallado de estos temas, aunque son textos un poco antiguos. 


En primer lugar, ofrecemos una introducción histórica para situar al lector en contexto. No pretendemos ser exhaustivos, sino más bien dar un panorama general de cómo el Álgebra Lineal y la Teoría de Gráficas han ido creciendo y coincidiendo a lo largo del tiempo.


El capítulo 1 está dedicado a establecer los conceptos principales y las definiciones de las que haremos uso en este trabajo.  Se divide en dos secciones, una dedicada al Álgebra Lineal y la otra dedicada a la Teoría de Gráficas. La mayoría de los teoremas en este capítulo se enuncian sin demostración y damos las referencias en las que el lector puede encontrar más información al respecto.


En el capítulo 2 estudiamos las propiedades de los conjuntos de corte y las gráficas pares. Este es el más largo y más ``pesado” de los demás capítulos porque es, prácticamente, el soporte teórico del resto. Casi todos los teoremas del capítulo 3, por ejemplo, son una consecuencia directa de los que se enunciaron en el capítulo2 y, por lo tanto, facilitan mucho las demostraciones. 


El capítulo 3 es mi favorito y es el más ``profundo” porque retomamos las ideas del capítulo anterior y las consideramos desde el punto de visto algebraico. Es aquí donde aparecen de forma natural los espacios vectoriales de una gráfica y de una digráfica. Tendemos el puente entre el mundo del Álgebra Lineal y el de la Teoría de Gráficas.  


Finalmente, la tesis se concluye con el capítulo 4, en el cual presentamos algunas de las consecuencias que han tenido las ideas del capítulo anterior. Las aplicaciones de este capítulo son tan bastas que fácilmente podría hacerse una tesis de cada uno. Así que hablamos de ellas de manera breve y sin ahondar tanto los aspectos técnicos.


Hemos incluido varias imágenes para explicar algunos conceptos y nuestros ejemplos terminan con el símbolo ``$\blacklozenge$" \hspace{1 mm} para separarlos del resto del texto. De igual modo, las demostraciones de los teoremas se terminan con ``\textbf{QED}", que abrevia la frase latina \textit{Quod Erat Demonstrandum} y significa ``que era lo que queríamos demostrar".

\begin{flushright}

Diciembre, 2020.

\end{flushright}