\chapter*{Prefacio}
   % \thispagestyle{empty}

\lettrine[lines=9] {\initfamily  \selectfont L}{a} \textit{Teoría Algebraica de Gráficas} es la rama de las matemáticas que se ocupa de aplicar los métodos algebraicos en problemas que conciernen a las gráficas. Estos métodos pueden incluir, por ejemplo, el uso de la teoría de grupos para estudiar invariantes en gráficas. Además, las técnicas del Álgebra Lineal han resultado ser muy fructíferas en la Teoría de Gráficas. En la presente tesis nos enfocamos en estudiar la relación entre estas dos amplias ramas de las matemáticas. Concretamente, en el análisis de los diferentes espacios vectoriales que surgen de las gráficas y digráficas y sus matrices de incidencia. Es preciso mencionar que nuestro texto base es el libro escrito por J. A. Bondy y U. S. R. Murty titulado \textit{Graph Theory} \cite{Bondy}. Además de moderno, encontramos conveniente  su notación y su terminología. El lector podrá hallar en \cite{Deo} y \cite{Seshu} un desarrollo mucho más profundo de estos temas, aunque son textos un poco antiguos.

Así, pues, ofrecemos en primer lugar una introducción histórica para situar al lector en contexto. No pretendemos ser exhaustivos, si no más bien dar un panorama general de cómo el Álgebra Lineal y la Teoría de Gráficas han ido creciendo y coincidiendo a lo largo del tiempo.

El capítulo 1 está dedicado a establecer los conceptos principales y las definiciones de las que haremos uso en este trabajo. En el capítulo 2 estudiamos las propiedades de los conjuntos de corte y las gráficas pares. En el capítulo 3 retomamos las ideas del capítulo anterior y las consideramos desde el punto de visto algebraico. Es aquí donde aparecen de forma natural los espacios vectoriales de una gráfica y de una digráfica. Concluimos con el capítulo 4, en el cual presentamos brevemente ciertas consecuencias que ha tenido el impacto del Álgebra Lineal en la Teoría de Gráficas y en otras ramas de las Matemáticas.
