\chapter*{Agradecimientos}
   % \thispagestyle{empty}
   
   \lettrine[lines=9] {\initfamily  \selectfont G}{racias},\textit{ Señor, porque sé que existes, porque en el mundo y en la vida estás presente Tú. Te doy gracias por cuanto soy. Cuanto puedo y cuanto recibo es un regalo tuyo}. Gracias porque desde mi nacimiento Tus ojos no se han apartado de mí. Hoy estoy aquí gracias a Ti. Gracias, Señor, por concederme la oportunidad de estudiar Matemáticas y a través de ella admirar Tu creación. Gracias, Poder Superior, por acompañarme en los años de mi carrera universitaria y por permitirme terminar esta tesis para gloria Tuya. Aunque muchas veces atravesé el valle de sombras, nunca soltaste mi mano. Has sido testigo de mis risas y lágrimas, de mi felicidad y mi tristeza y del esfuerzo que me tomó concluir mi carrera y este trabajo. Las muchas palabras jamás serán suficientes, Señor, para comunicarte lo eternamente que estoy agradecido contigo.


\textit{A Dios le pedí fuerzas para grandes logros: me hizo débil para aprender y humilde a obedecer. Pedí salud para hacer cosas grandes: me dio enfermedad para poder hacer cosas buenas. Pedí riqueza para poder ser feliz: me dio debilidad para sentir necesidad de Dios. Pedí todo para poder disfrutar de la vida: me concedió vida para poder disfrutar de todo. Pedí lujos y fama: me concedió amigos y amor. A pesar de mí mismo, las peticiones que no hice me fueron concedidas. ¡Dios mío! Entre los hombres soy el más afortunado. ¡Gracias Dios mío!}



A mi mamá, Margarita. Gracias por darme la vida y soportar el peso que implicó mantener a dos niños tú sola y siendo muy joven. Gracias, madre, porque por ti soy quien soy. Por desvelarte por mi hermano y por mí. Por esos desayunos, esas comidas, esas cenas, esos juguetes. Esas medicinas que nos dabas cuando enfermábamos. Esos consejos que nos dabas cuando más lo necesitábamos. Por esas sonrisas y esos regaños que nos han forjado. Por los valores que nos diste. Por cada minuto de tu vida que nos otorgaste, aunque eso implicara a veces privarte de ciertas cosas. Por siempre, agradecido contigo. Hoy tienes la dicha de decir que tus dos hijos están graduados. Te amo.


A mi hermano Abraham. Te admiro por tu valor y el ánimo constante que tienes por superar las adversidades, enfrentar la vida y, sin conformarte, buscar estar mejor cada día. Gracias porque hoy tú, nuestra mamá y yo estamos juntos, y juntos hemos salido adelante. 
A mi abuelita, Mercedes y a mi abuelito Alfonso. A mis primos hermanos Janeth, Susy e Iker.  A mi tía Susana, mi tío Poncho y mi tía Sandra. Crecí con ustedes y son parte de mí. Cada uno ha impactado en mi vida y me han dejado innumerables enseñanzas. Los amo.


A mis mascotas: Chiquis, Benita, Pulga, Pipa y Lira. Porque me enseñaron el amor y el respeto hacia la vida de los animales.


A mi tía Elena y a mi tía Maricela. Porque cambiaron mi vida para siempre. Y por eso siempre estaré en agradecido con ustedes. Gracias por el infinito apoyo que nos brindaron cuando más lo necesitábamos, por confiar en nosotros y por siempre tenernos en cuenta. Las amo.


Recuerdo mucho que un día, cuando yo era niño, mi mamá me comentó que a veces Dios nos manda ángeles para ayudarnos en diversas situaciones. Me quedé con esa enseñanza y la sigo llevando en mi corazón. A través de los años he sido bendecido por la presencia de personas muy especiales para mí. Estos son mis ángeles,a  los que quiero agradecer:


A Angélica Reyes. Desde que te conocí, no fui el mismo. Gracias por enseñarme a ver las situaciones de la vida de forma diferente. Gracias por compartir tu vida conmigo. Por tu amor, por tu comprensión y por tu paciencia. Gracias a tu familia: a tu papá y a tu mamá y a hermano, porque siempre me han recibido bien, porque cada que platico y convivo con ellos me llevo una enseñanza. Agradezco porque son un ejemplo y son una familia fuertemente unida, que afrontan las adversidades juntos, sin amargarse, sin desanimarse, más bien ayudando aún más a los demás. Te admiro por el bello ser humano que eres. Admiro tu fuerza, tu disciplina y tu constancia. Admiro que eres de las personas más capaces que he conocido y sé que llegarás lejos y tendrás éxito donde quiere que estés y te propongas estar. De todos los años que he estado en la Universidad, eres lo mejor que me ha sucedido. Siempre estaré agradecido contigo. Soy bendecido y afortunado por conocerte. Te quiero mucho.


Al Dr. Gilberto Calvillo. Las palabras nunca me alcanzarán para expresarle mi gratitud. No se imagina lo mucho que me ha ayudado y lo mucho que he aprendido con usted. Conocerlo cambió mi vida para siempre. Usted ha sido como mi guía estos años. Admiro grandemente el ser humano que es. Su amor, su amabilidad y la forma que tiene usted de compartir lo que tiene sin esperar nada a cambio. Que usted busca el ayudar sólo por ayudar. Estoy eternamente agradecido con usted. Sé que algún día a mí me tocará ayudar a alguien, así como usted lo ha hecho conmigo. Y cuando ese día llegue, estaré listo y responderé con gratitud, acordándome y llevando en mi mente el cómo usted lo hizo conmigo. Usted es mi ejemplo. Gracias Dr. Calvillo, por todo. Gracias, gracias, gracias.


A Miguel Giles y su familia. Gracias, don Miguel, por el incalculable apoyo que me dio. Gracias a eso, pude terminar mi carrera a tiempo. Gracias por cada enseñanza y cada plática que tuve con usted. Gracias por orientarme y compartir su tiempo. Gracias por haberme ayudado cuando más lo necesitaba. Gracias por el apoyo y cariño que siempre recibí de su familia, de su esposa y de sus hijos.


A Alejandra Camacho y a Esperanza Abarca. Gracias Pera, por la inmensa ayuda que me brindó. Por haberme apoyado durante mi carrera. Por sus consejos, por su tiempo, por las comidas que me dio. Gracias por cada detalle que tuvo conmigo. Gracias por recibirme en su casa de la mejor manera. Todo eso lo llevo en mi corazón por siempre. Gracias, Ale, por enseñarme que \textit{a pesar de que conozcamos algo muy malo dentro de nosotros, eso no define lo que en realidad somos}. Gracias por cada momento de tu tiempo y por cada enseñanza y el cariño que me brindaste. También le agradezco a toda tu familia: a tu papá y a tu hermano, a tus primos y primas, tus tíos y tías y a tu abuelita, porque cada uno dejó algo en mi corazón, y cada uno siempre me apoyó a su manera. Estoy por siempre agradecido con ustedes. 


A Arturo Calvillo. Gracias por tu amistad y por tu compañía y por el gran ser humano que eres. Por esas charlas que hemos tenido juntos. Te admiro mucho y agradezco grandemente el apoyo que me has brindado desde que nos conocimos.


A Felipe Spínola. Gracias por las enseñanzas que me dio y por sus consejos que me acompañan. Gracias por las comidas que compartimos y por escucharme. Con usted y con Arturo, cunca me sentí sólo en Coyoacán. Siempre conté con usted. Le agradezco infinitamente por todo.


A Gaby Campero. Desde mi primer día de clases en la Universidad no he dejado de aprender de ti. Gracias por mostrarme lo bello de las Matemáticas, de la Teoría de Conjuntos y la Lógica.  Gracias por inculcar en mí la preocupación por el rigor matemático. Gracias por darme el honor de ser ayudante tuyo por un año completo. Y gracias por ser un ejemplo de vida y de ser humano, que no decae ante las adversidades. 


A Leonel Rito. Eres de mis más grandes amigos. Siempre he podido contar contigo, no importa lo que sea. Te admiro como matemático y como ser humano. Gracias por escucharme, por hacerme compañía. Por esas veces que estudiábamos juntos en la sala de maestros o en el Instituto de Matemáticas de Cuernavaca. Por permitirme conocer a tu familia y por cada consejo que me diste. Sobre todo, por el privilegio de ser tu ayudante por un año. Esto fue un acontecimiento que me cambió la vida, por las experiencias que tuve, por la gente que conocí gracias a ti.   


A Ulises Cendejas, a Jenny Zúñiga y a Liz Fernández. Siempre que lo necesité, ahí estaban conmigo. Siempre tuve su apoyo. Cuando no tenía dinero, cuando no tenía dónde pasar la noche, ahí estaban ustedes. Nunca me sentí sólo en la Universidad, porque sabía que podía contar con ustedes. Gracias por cada experiencia, por cada reunión que hicimos, por cada charla, por hacer que se me salieran las lágrimas de la risa que me provocaban, por pasar un semestre completo haciendo ejercicio. Por invitarme cada uno a sus actividades. Gracias por todo.


A mis amigos del cubículo de Gaby, Ana Lissette, Róger López, Mariana Garduño, Tonatiuh Velázquez, Manuel Zúñiga y Yanh Vissuet. Siempre me gustó ir con ustedes a echar relajo. Era relajante. Gracias por esas reuniones, por esos cumpleaños. Porque fuimos compañeros de trabajo, también fueron mis ayudantes y mis compañeros de clase. Por cada examen que presentamos juntos y, posteriormente, cada examen que calificamos juntos. Gracias porque hoy los considero mis amigos. 
 
A mis alumnos de Álgebra y Cálculo, Nahum Vega, Marcos Baltasar, Fer Garnica, Diana Sandoval, Gabriel Chino, Jonathan Flores, Clarissa Barreto, Vania Villegas, Amacalli Pelcastre, Itzanami Karina, Aranza Ortiz, Leobardo Enríquez, Mónica Ávila, Brenda Guízar, Daniel Cruz y Yair Franco. Gracias por permitirme darles clase y por tenerme paciencia. Gracias porque ustedes hacían divertidas las clases y yo disfrutaba enseñarles. Gracias porque todos ustedes los considero mis amigos y les deseo éxito en sus carreras y bienestar en sus vidas.



Al Dr. Jorge Rivera Noriega, por escucharme y orientarme cuando no estaba seguro de estudiar Matemáticas. Porque podía ir a su cubículo y usted siempre tenía tiempo para mí. Gracias por cada experiencia que me compartió. Gracias Dr. Jorge.


A Guadalupe Santiago, porque fuiste tú quien me sembró el amor por las Matemáticas. Gracias por esos entrenamientos que me diste. Gracias Lupito, por ser mi profesor y mi amigo. 


A mi profesor de la primaria, el maestro Christian. Por las enseñanzas que me dio cuando era niño.


A mis profesores de la secundaria del Colegio Latinoamericano de Tabasco. en especial al maestro Rosario, al maestro Miguel Ángel y a la maestra Ernestina. Los recuerdo con mucho cariño. 


A mis profesores de la secundaria Lic. Manuel Sánchez Mármol. En especial al profesor Juan Arturo, al profesor Wilbert y al profesor Collado. Recuerdo con mucho cariño a todo el personal de esta escuela. Aquí fue mi ``despertar intelectual” y me agradaba ir de vez en cuando a la biblioteca a leer.


A mis profesores de la preparatoria COBATAB no 6. Todos en especial fueron grandes profesores para mí. Al profesor Reynolds, al profesor Poot y al director Heredia los recuerdo con cariño.


A mis profesores de la preparatoria del Centro Universitario Americano del Estado de Morelos. En especial al profesor Alcalá, al profesor Ernesto, a la maestra Graciela, al profesor Raúl Tapia, a la maestra Liliana, al profesor Lot, a la maestra Bertha, a la maestra Rebeca y a la maestra Jessica, a Chivis, a Isita, a la señora Edith. Gracias por todo su apoyo y por la paciencia de enseñarme.


A mis profesores de la Facultad de Ciencias. Héctor Méndez, Óscar Palmas, Silvestre Cárdenas, María del Rocío y Loiret Alejandría. Gracias por haberme enseñado de la manera que lo hicieron porque son mi ejemplo para dar clases.


A la profesora Laura Pastrana. Por enseñarme por vez primera el hermoso mundo de las gráficas. Siempre me dio su apoyo. Y cada vez que necesitaba ayuda u orientación, yo sabía que podía recurrir a usted. Tengo el honor de haber sido su alumno. Gracias por todo.


A mis profesores del Instituto de Matemáticas de Ciudad Universitaria y de Cuernavaca, Dr. Erick Treviño, Dr. Carlos Villegas, Dr. Aubin Arroyo, Dr. Carlos Cabrera, Dr. José Luis Cisneros, Dr. Carlos Prieto, Dra. Laura Ortiz y el Dr. Ernesto Rosales por su apoyo siempre incondicional hacia mí.


A mis ayudantes, Mario Nolasco, Felipe Hernández, Germán Benítez, Jorge Alonso, Rodrigo Domínguez, Jesús Palma y Ricardo Guerrero, por su amistad y transmitirme su conocimiento. 


A mis amigos del Instituto de Matemáticas de Ciudad Universitaria y de Cuernavaca, Lucinda Serna, Erich Ulises, Jessica Morales, Karla Isayuvi, Daniela Hernández y Melissa Ponce.


A mis sinodales. Dr. Jesús López, Dr. Pablo Barrera, Dr. David Romero, mi más grande aprecio y reconocimiento para ustedes. Muchas gracias por aceptar ser parte de mi jurado y leer este trabajo y tomarse el tiempo de encontrar detalles para corregirlo. Gracias por su paciencia, por su amabilidad y por siempre apoyarme cuando lo necesitaba. Muchísimas gracias.


Al pastor Vitelio Toledo, por el apoyo que nos brindó cuando más lo necesitábamos. Siempre contamos con usted y siempre nos ofreció su ayuda. Gracias Vite, por absolutamente todo. 


A Padmabandhu, por enseñarme el camino de la meditación y el budismo. Gracias por enseñarme a vivir con serenidad, en el aquí y el ahora y, poco a poco, sin apegos. Gracias por todo Padma. 


A mi madrina Fran y a mi padrino Rey, por acompañarme en el proceso de conocerme a mí mismo.


A la psicóloga Martha Chávez, por salvarme la vida. Por sacarme del hoyo en el que me estaba hundiendo. Gracias por hacerme rendirme y despojarme de mi soberbia, de mi ira y de mi ego. Por abrir los ojos y nacer a una nueva vida. Gracias por enseñarme el camino de la aceptación y amarme tal y como soy. Por enseñarme a ver a mi prójimo con amor, como ser humano. Por enseñarme a perdonar y a soltar. Por tirar el peso del pasado y avanzar. Por enseñarme a que la vida es hermosa tal y como es. A dejar de renegar del pasado y de las circunstancias. A ver hacia adelante, aceptando este mundo tal cual es y no como yo creo que debería ser, así como hizo Cristo en la Tierra. Gracias por enseñarme a vivir un día a la vez y darme cuenta que merezco ser feliz y merezco estar bien. Gracias Martha, por ser mi luz y mi guía. Te aprecio con todo mi corazón.


A la Universidad Nacional Autónoma de México. Es un orgullo ser parte de ti. Desde que estaba en la secundaria, cuando tenía 14 o 15 años, siempre quise pertenecer a la Universidad. Y hoy estoy aquí, a punto de titularme y terminando mi segundo semestre de la maestría en tus Institutos. Gracias, Universidad, por todo lo que me has dado, porque he viajado, he andado y he conocido mucha gente gracias a ti. Estaré eternamente agradecido contigo y, cada que tenga la oportunidad, trataré de devolverte lo mucho que me diste, sirviendo a los demás con los conocimientos que me otorgaste.

\begin{flushright}
\textit{Por mi raza hablará el espíritu}.

Diciembre, 2020.

\end{flushright}










